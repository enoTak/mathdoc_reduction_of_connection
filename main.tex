\documentclass[b5paper]{article}


\usepackage{subfiles}
\usepackage{amsmath, amsfonts}
\usepackage{amsthm}
\usepackage{tikz-cd} 


\newcommand{\field}{\mathcal{K}}


\newcommand{\worddef}{\textrm}


\newcommand{\frakg}{\mathfrak{g}}
\newcommand{\frakh}{\mathfrak{h}}


\newtheorem{proposition}{Proposition}
\newtheorem{theorem}{Theorem}
\newtheorem{definition}{Definition}


\title{Reduction of connections on vector bundles}
\date{}


\begin{document}


  \maketitle


  \begin{definition}
    Let $\pi_P: P \to M$ be a principal $G$-bundle and
    $\pi_Q: Q \to M$ be a principal $H$-bundle.
    We say $Q$ is a {\rm reduction} of $P$ if there is an embedding $\iota_Q: Q \to P$ and
    an injective homomorphism $\iota_H: H \to Q$ such that it holds that
    $\iota_Q(\eta h) = \iota_Q(\eta) \iota_H(h)$ for any $\eta \in Q$ and $h \in H$.
  \end{definition}


  Consider principal $G$-bundle $\pi_P: P \to M$, principal $H$-bundle $\pi_Q: Q \to M$,
  and associated bundle $P \times_\rho V$ for
  a representation $\rho: G \to GL(V)$.
  If there is an homomorphism $\iota_H: H \to G$,
  we can another associated bundle $Q \times_\sigma V$ for
  the restriction representation for $\sigma = \rho \circ \iota_H$ of $H$.


  \begin{proposition}
    If $Q$ is a reduction of $P$, then there is a natural isomorphism between
    $P \times_\rho V$ and $Q \times_\sigma V$.
  \end{proposition}


  \begin{proof}
    Consider a map $Q \times V \to P \times V$, $(\eta, v) \mapsto (\iota_Q(\eta), v)$.
    Then it holds that if $(\eta h, v) \sim (\eta, \sigma(h)v)$ then
    \begin{align*}
      (\iota_Q(\eta h), v) &= (\iota_Q(\eta) \iota_H(h), v) \\
                           &\sim (\iota_Q(\eta) , \rho(\iota_H(h))v) \\
                           &= (\iota_Q(\eta) , \sigma(h)v).
    \end{align*}
    Hence the map $\varphi: Q \times_\sigma V \to P \times_\rho V$, 
    $\eta \times_\sigma v \mapsto \iota_Q(\eta) \times_\rho v$ is well-defined.
    We want to show that $\varphi$ is a desired isomorphim.

    We construct well-handled local trivializations of our bundles.
    First we consider $Q$ and choose some local trivialization
    $\phi^Q: Q|_U \to U \times H$.
    Then we have a section $q: U \to Q|_U$ such that
    $\phi^Q \circ q (y) = (y, e)$.
    Using the section $q$, we can denote
    $\phi^Q(\eta) = (y, h)$ where $y = \pi_Q(\eta)$ and $\eta = q(y) h$.

    We construct $\phi^P: P|_U \to U \times G$
    which satisfies the following commutative diagram:
    \[
      \begin{tikzcd}
      P|_U \arrow[r, "\phi^P"] 
      & U \times G\\
      Q|_U \arrow[u,"\iota_Q"] \arrow[r, "\phi^Q"] 
      & U \times H \arrow[u, "id \times \iota_H"'] \\
      U \arrow[u, "q"] \arrow[ur, "id \times \{e\}"']
      \end{tikzcd}
    \]
    Let $p = \iota_Q \circ q$ and define $\phi^P: P|_U \to U \times G$ by
    $\phi^P(\xi) = (x, g)$ where $x = \pi_P(\xi)$ and $\xi = p(x) g$.
    Then it holds that for $\eta = q(y)h \in Q|_U$
    \begin{align*}
      \iota_Q(\eta) &= \iota_Q(q(y)h) = \iota(q(y)) \iota_H(h) \\
                    &= p(y) \iota_H(h) \\
      \phi^P(\iota_Q(\eta)) &= (y, \iota_H(h)).
    \end{align*}
    This shows the commutativity of upper rectangle.
    The commutativity of lower triangle follows by the definition of $q$.
    Note that $\phi^P \circ p (x) = (x, \iota_H(e)) = (x, e)$ for $x \in U$.


    Next we construct local trivializations for $E^P$ and $E^Q$.
    Define $\psi^Q(q(y) \times_\sigma v) = v$ and
    $\psi^P(p(x) \times_\sigma u) = u$.
    Then trivially it holds that the following commutative diagram: 
    \[
      \begin{tikzcd}
      E^P|_U \arrow[r, "\psi^P"] 
      & U \times V\\
      E^Q|_U \arrow[u,"\varphi"] \arrow[r, "\psi^Q"] 
      & U \times V \arrow[u, equal]
      \end{tikzcd}
    \]
    This implies that $¥varphi$ is an isomorphism from $E^Q$ to $E^P$.

  \end{proof}

  For a convinience in the following section,
  we re-state the exietence of well-handled local trivializations of our bundles:

  \begin{proposition}
    For a local trivialization $\phi^Q: Q|_U \to U \times H$,
    difine 
    \begin{itemize}
      \item a section $q: U \to Q|_U$ by $\phi^Q \circ q (y) = (y, e)$;
      \item a section $p: U \to P|_U$ by $p = \iota_Q \circ q$;
      \item a local trivialization $\phi^P: P|_U \to U \times G$ of $P$ by 
        $\phi^P(\xi) = (x, g)$ where $x = \pi_P(\xi)$ and $\xi = p(x) g$;
      \item a local trivialization $\psi^Q: E^Q|_U \to U \times V$ of $E^Q$ by
        $\psi^Q(q(y) \times_\sigma v) = v$;
      \item a local trivialization $\psi^P: E^P|_U \to U \times V$ of $E^P$ by
        $\psi^P(p(x) \times_\sigma u) = u$.
    \end{itemize}
    Then it holds the following commutative diagrams:
    \[
      \begin{tikzcd}
      P|_U \arrow[r, "\phi^P"] 
      & U \times G\\
      Q|_U \arrow[u,"\iota_Q"] \arrow[r, "\phi^Q"] 
      & U \times H \arrow[u, "id \times \iota_H"']
      \end{tikzcd} %
      \begin{tikzcd}
      E^P|_U \arrow[r, "\psi^P"] 
      & U \times V \\
      E^Q|_U \arrow[u,"\varphi"] \arrow[r, "\psi^Q"] 
      & U \times V \arrow[u, equal]
      \end{tikzcd}
    \]
  \end{proposition}


  \begin{definition}
    Let $\pi_P: P \to M$ be a principal $G$-bundle and
    $\pi_Q: Q \to M$ be a principal $H$-bundle which is a reduction of $P$.
    We call a connection form $\theta \in \Omega^1(P; \frakg)$ is {\rm reducible to $Q$}
    if $(\ker \theta)_{\iota_Q(\eta)} \subset (\iota_Q)_{*\eta}(T_\eta Q)$
    for any $\eta \in Q$.
  \end{definition}

  \begin{theorem}
    Consider a (real or complex) Hermitian verctor bundle $(E, h)$,
    frame bandle $P$ of $E$, and frame bandle $Q$ of $(E, h$).
    For a connection form $\theta \in \Omega^1(P; \frakg)$ 
    the followings are equivalent:
    \begin{itemize}
      \item the induced connection $\nabla^\theta$ on $E$ is $h$-preserving;
      \item $\theta$ is reducible to $Q$.
    \end{itemize}

  \end{theorem}

\end{document}